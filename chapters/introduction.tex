\section{Introduction}
Entity Matching (EM), the task of identifying records that refer to the 
same real-world entity, is a cornerstone of modern data integration 
pipelines. As organizations increasingly migrate to cloud-based data 
ecosystems, the necessity of consolidating heterogeneous data sources 
has become paramount. While traditional EM approaches relied on 
heuristic rules or manual feature engineering, the state-of-the-art 
has shifted toward sophisticated deep learning architectures. 
Modern frameworks increasingly leverage Transformer-based models, 
such as BERT, which utilize self-attention mechanisms to achieve 
unprecedented accuracy in identifying duplicates across noisy and 
unstructured datasets.

Despite these performance gains, these models operate as "black boxes." 
Their decision-making processes are embedded within high-dimensional 
vector spaces and complex non-linear transformations, offering no 
inherent transparency. In industrial applications, this lack of 
interpretability poses significant risks, particularly regarding 
data governance, accountability, and the "right to explanation" 
under regulatory frameworks like the GDPR. As highlighted in a 
recent overview by Araújo et al. \cite{Araujo2025}, ensuring 
fairness and transparency in EM is critical for the deployment of 
trustworthy data integration systems.



This research addresses the transparency-performance tradeoff by 
evaluating post-hoc explainability techniques applied to a fixed 
Transformer-based EM model. We investigate the efficacy of 
\textbf{Lemon} (Local Interpretable Model-agnostic Explanations), 
\textbf{Integrated Gradients} (a gradient-based attribution method 
implemented via Captum), and \textbf{Adversarial Counterfactuals} 
(generated through the TextAttack framework). 

By benchmarking these methods against established EM datasets, we 
provide a comparative analysis of their \textit{fidelity}, 
\textit{stability}, and \textit{usability}. Our goal is to 
establish a framework that not only identifies duplicates but 
also justifies the underlying logic through actionable insights, 
thereby facilitating system debugging and increasing stakeholder 
trust in automated master data management.